\documentclass[10pt,a4paper]{scrartcl}
\usepackage[utf8]{inputenc}
\usepackage[german]{babel}
\usepackage[T1]{fontenc}
\usepackage{amsmath}
\usepackage{amsfonts}
\usepackage{amssymb}
\usepackage[left=2cm,right=2cm,top=1cm,bottom=1cm]{geometry}

\title{Reinigungsplan}
\subtitle{Eröffnungsveranstalltung Mitmachwerkstatt am 15.08.2020}
\date{\vspace{-8ex}}

\begin{document}
\maketitle

% table row spacing
\renewcommand{\arraystretch}{2.0}

\section{Desinfektion von Gegenständen und Oberflächen}
Alle \textbf{zwei Stunden} müssen häufig berührte Gegenstände und Oberlächen desinfiziert werden. Dazu wird ein mit
Desinfektionsmittel getränktes Tuch benutzt und die Oberflächen und Gegenstände gereinigt.

\subsection{Im Raum}
\begin{itemize}
\item Türgriffe Eingangstür
\item Türgriffe Duchgangstür zur Toilette
\item Exponate
\end{itemize}

\subsection{Aussenbereich}
\begin{itemize}
\item Tische und Bänke
\item Spendenkasse
\item Kühlschranktür
\item Flaschenöffner
\item Kulischreiber (Adressdaten der Teilnehmer)
\end{itemize}


\begin{tabular}{ | r | p{15cm} |}
\hline
\textbf{Uhrzeit} & \textbf{Name / Unterschrift} \\ \hline
11:00h & \\ \hline
13:00h & \\ \hline
15:00h & \\ \hline
17:00h & \\ \hline
18:00h & \\ \hline
\end{tabular}

\vspace{4ex}

\section{Desinfektion der Toiletten}
Alle \textbf{zwei Stunden} müssen die Toiletten desinfiziert werden.
Dabei wird mit einer Sprühflasche das Desinfektionsmittel aufgetragen.
Papierhandtücher und Handwaschmittel der Toiletten muss kontrolliert und ggf.\ ersetzt werden.
\begin{itemize}
\item Türgriffe Toiletten
\item Spülung
\item Toilettensitz
\end{itemize}

\begin{tabular}{ | r | p{15cm} |}
\hline
\textbf{Uhrzeit} & \textbf{Name / Unterschrift} \\ \hline
11:00h & \\ \hline
13:00h & \\ \hline
15:00h & \\ \hline
17:00h & \\ \hline
18:00h & \\ \hline
\end{tabular}

\end{document}
