\documentclass[10pt,a4paper]{scrartcl}
\usepackage[utf8]{inputenc}
\usepackage[german]{babel}
\usepackage[T1]{fontenc}
\usepackage{amsmath}
\usepackage{amsfonts}
\usepackage{amssymb}
\usepackage[left=2cm,right=2cm,top=2cm,bottom=2cm]{geometry}

\title{Hygienekonzept}
\subtitle{Eröffnungsveranstaltung Mitmachwerkstatt am 15.08.2020}
\date{\vspace{-8ex}}

\begin{document}

\maketitle
\section{Vorbemerkung}
Wir erfüllen die Definition einer Veranstaltung nach CoronaVO BW §10 Abs. 6.
Da wir externe Gäste haben handelt es sich um eine öffentliche Veranstaltung, damit brauchen wir ein Hygienekonzept.




\section{Allgemeines}
\subsection{Vereinsmitglieder}
\begin{itemize}
\item Alle anwesenden Vereinsmitglieder werden über die Maßnahmen des Hygienekonzepts informiert.
\item Vereinsmitglieder müssen auf die Einhaltung des Hygienekonzepts achten und ggf.\ durchsetzen.
\end{itemize}


\subsection{Gäste}
\begin{itemize}
\item alle Gäste werden beim Empfang über die Schutz- und Hygieneregeln informiert.
\item Erfassung der Kontaktdaten (siehe `Datenerhebung der Teilnehmer')
\item Teilnahmeverbots von Risikopersonen (siehe `Teilnameverbots von Risikogruppen')
\end{itemize}

\subsection{Beschilderung}
\begin{itemize}
\item Hinweise auf die Regeln zum Abstand und Mundschutz im Raum werden am Eingang gut sichtbar angebracht.
\item In den Sanitärräumen befinden sich Hinweise zum Händewaschen und zur Handdesinfektion.
\end{itemize}




\section{Allgemeine Hygiene- und Distanzregeln}
\begin{itemize}
\item Grundsätzlich gilt das Einhalten des Mindestabstands (1,5 Meter) in allen Bereichen.
\item In Räumen gilt die Maskenpflicht wobei Mund und Nase bedeckt sein müssen.
\item Im Raum sind nur Führungen mit max.\ drei Gästen und einem Vereinsmitglied erlaubt.
\item Desinfektion von Gegenständen und Oberflächen welche häufig von Personen berührt werden (siehe extra `Reinigungsplan').
\item Der Raum muss nach jeder Führung gelüftet werden.
\item Gästen werden die Hände beim kommen durch ein Vereinsmitglied desinfiziert.
\item Beachten der Hust- und Nies-Etikette (Armbeuge oder Einmal-Taschentuch)
\end{itemize}


\pagebreak

\section{Teilnahmeverbots von Risikopersonen}
Sollte eine Frage mit `Ja' beantwortet werden, muss die Person von der Veranstaltung ausgeschlossen werden.
\begin{itemize}
  \item Gab es innerhalb der vergangenen 14 Tage direkten Kontakt zu einem bestätigten Corona-Fall?
  \item Bestand in den vergangenen vier Wochen die behördliche Anordnung einer Quarantäne im Zusammenhang mit Corona?
  \item Wurde innerhalb der vergangenen 14 Tage eine Region mit erhöhter Anzahl an positiven Corona-Fällen besucht?
  \item Liegen aktuell und / oder lagen in den vergangenen 14 Tagen eines der folgenden Sympthome vor?
  \begin{itemize}
    \item Fieber (über 38 Grad)
    \item (Trockener) Husten
    \item Atemnot
    \item Geschmacks- und / oder Riechstörungen
    \item Halsschmerzen
  \end{itemize}
\end{itemize}




\section{Datenerhebung der Teilnehmer}
Erfassung der Kontaktdaten (Vor- und Nachnamen, Adresse, Telefonnummer / Mail, Zeitraum der Anwesenheit).
\begin{itemize}
  \item Daten werden nur zur Nachverfolgung mgl.\ Infektionsketten vom Gesundheitsamt eingesehen.
  \item Kontaktdaten werden auf Zetteln geschrieben und in eine Box geworfen.
  \item Die Box wird vom Vorstand in Verwahrung genommen.
  \item Der Inhalt der Box wird 4 Wochen nach dem Event vernichtet.
\end{itemize}


\end{document}
